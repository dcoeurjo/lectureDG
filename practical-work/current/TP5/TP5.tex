\documentclass[a4paper, 11pt]{article}

\usepackage{hyperref}
\voffset -0cm
\hoffset 0.0cm
\textheight 23cm
\textwidth 16cm
\topmargin 0.0cm
\oddsidemargin 0.0cm
\evensidemargin 0.0cm

\usepackage{epsfig}
\usepackage{setspace}
\usepackage{fancyheadings}
\usepackage{amsmath}
\usepackage{amssymb}
\usepackage{graphicx}
\usepackage{url}
\usepackage{verbatim}
\title{}
\author{}
\date{}

\begin{document}

\begin{center}
	\LARGE \textbf{TP5: Convex Hulls and Digital Convex Hulls}
\end{center}

\bigskip
\par In this TP, the idea is to implement a convex hull algorithm and to experiment its complexity (number of edges) on digital objects.


\section*{Preliminaries \rm Writing svg files}

In this lab, you will need to write images of points and lines. A practical way is to write svg files, which are text files using an xml markup language. They can be directly opened, for instance, in your browser. A simple svg file writing some text and plotting one dot and one line looks like:

\begin{verbatim} 
<svg width="512" height="512"  xmlns="http://www.w3.org/2000/svg">
  <line x1="15" y1="5" x2="500" y2="95" stroke="red" />
  <circle cx="256" cy="256" r="5" fill="blue" />
  <text x="210" y="60">Text</text>
</svg>
\end{verbatim} 
% -------------------------------------------------------
\section*{Exercise 1 \rm Orientation predicate}

{\bf Question:} Implement the $Orientation(p,q,r)$ predicate as discussed in the lecture.
	\begin{itemize}
	\item To avoid numerical issues, we encourage you to consider points in $\mathbb{Z}^2$ on a digital domain (instead of $\mathbb{R}^2$). 
	\item To display a line in \texttt{DGtal} between \texttt{Points} $p$ and $q$, you can use the following method of the \texttt{Board} class: \texttt{board.drawLine( p[0], p[1], q[0], q[1]);}
	\end{itemize}

{\bf Question:} Test the $Orientation$ predicate with an implementation of the segment-segment intersection detection (cf lecture). Experiment your intersection test on all cases (regular intersection, alignement, no intersection, intersection point is a vertex, ...).


\bigskip
\bigskip
\bigskip

\par The rest of the TP focuses on the implementation and the experimentation of a \emph{convex hull algorithm}, namely, Graham's scan algorithm. At the end we would like you:
	\begin{itemize}
	\item To test the convex hull construction on point sets defined by the digitization (at a given resolution $h$) of a disc defined as a digital set. You will write an svg file to display your result.
	\item To plot (using \texttt{gnuplot} or svg) the number of edges when $h\rightarrow 0$ in log-scale. The aim is to observe the $N^{2/3}$ behavior we discussed in the lecture for convex hull in $N\times N$ domains.
	\end{itemize}


\par Graham's scan implementation on the point set ($O(n.logn)$) should only use the $Orientation(p,q,r)$ predicate and point coordinate comparisons.


\section*{Exercise 2 - \rm Convex hull}

\noindent {\bf Questions:}
\begin{itemize}
	\item Implement the Graham's scan algorithm as described in the lecture:
		\begin{itemize}
		\item First, sort the points by polar angle (cf the \texttt{qsort} C function man page  or the C++ \texttt{std::sort} to do the sort). As discussed in the lecture, you would just have to replace the comparison function/functor by the Orientation predicate with fixed $p_0$.
		\item The second step consists in a simple stack based removal (using for example \texttt{std::queue}).
		\end{itemize}
	\item Display the number of edges as a function of $h$ in a graph to observe the $N^{2/3}$ behavior.
\end{itemize}

\par Note that Graham's scan can be sped up just considering border point and not interior points (for experiments on digital shapes).



\end{document}
